\documentclass[twocolumn]{paper}
\usepackage{graphicx} % Required for inserting images
\usepackage[a4paper, left=2cm, right=2cm, top=3cm, bottom=3cm]{geometry}
\title{Processo de inicialização do USM}
\author{Iso }
\date{February 2025}

\begin{document}

\maketitle

\begin{abstract}
  Resuminho bem mais ou menos da inicialização do Ultra Spinner Magnetometer\\
  para implementar referencias corretas: Doi2bib.org!
\end{abstract}

\section{Informações importantes}
Ambos o \emph{Software e Hardware} são prototipos e ainda estao em desenvolvimento.\\
O maquinário é Bivolt, mas pode ter mal funcionamento no 110V.\\
Nunca fechar o prompt aberto junto do Software USM control.\\

\section{Inicialização}
Para inicializar o USM ligar os dois interruptores do amplificador(primeiro o de baixo, depois o de cima), em sequencia aumentar a frequencia do amplificador ao maximo.\\
Após isso, ja no computador inicializar o softawe do USM (USM control), realizar a calibração do equipamento, e abrir a pasta na qual os dados serao salvos.

\section{Calibragem}
Apos a inicialização do equipamento e abertura do software, conectamos o software ao hardware manualmente.\\
Para a calibração, colocamos uma amostra modelo disponibilizada pelo fornecedor no holder da maquina e tiramos as tres medidas iniciais no formato NRM, que o proprio software solicita.

\section{Tiragem de medidas}
Com o Software aberto e a calibragem ja realizada, iniciamos as medidas das amostras, o processo de Tiragem de medida no USM é identico ao que foi realizado na calibração do mesmo, a unica diferença no caso é>Após isso passamos a amostra para o segundo software, o AFDE (esse nao precisa necessariamente ser calibrado) que realiza a magnetização da amostra.

\end{document}
